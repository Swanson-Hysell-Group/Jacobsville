\documentclass[11pt,letterpaper]{article}
\usepackage{xcolor}
\usepackage{textcomp,marvosym}
\usepackage{amsmath,amssymb}
\usepackage[left]{lineno}
\usepackage{changepage}
\usepackage{rotating}
\usepackage{natbib}
\usepackage{setspace}
\usepackage{fancyhdr}
\usepackage{graphicx}
\usepackage{sidecap}
\usepackage{pdfpages}
\usepackage{longtable}
\usepackage{url}

\usepackage[aboveskip=1pt,labelfont=bf,labelsep=period,justification=raggedright,singlelinecheck=off]{caption}
\doublespacing

\raggedright
\textwidth = 6.5 in
\textheight = 8.25 in
\oddsidemargin = 0.0 in
\evensidemargin = 0.0 in
\topmargin = 0.0 in
\headheight = 0.0 in
\headsep = 0.5 in
\parskip = 0.1 in
\parindent = 0.2 in

\pagestyle{myheadings}
\pagestyle{fancy}
\fancyhf{}
\lhead{}
\rhead{\thepage}

\begin{document}

\begin{flushleft}

{\Large \textbf{Continuous motion from the Keweenawan track to the Grenville Loop: new paleomagnetic pole of the ca. 1 Ga Jacobsville sandstone}}
\\
\singlespacing
 Yiming Zhang\textsuperscript{1}, Nicholas Swanson-Hysell\textsuperscript{1}, Blake Hodgin\textsuperscript{1}\\
\bigskip
\textsuperscript{1} Department of Earth and Planetary Science, University of California, Berkeley, CA, USA\\

\smallskip
\end{flushleft}

%\linenumbers
%\pagestyle{empty}

\section*{ABSTRACT}


\section*{INTRODUCTION}

The extensive paleomagnetic studies of the well-preserved late Mesoproterozoic to Neoproterozoic rocks of North America have provided the central record for global paleogeography reconstruction during the assembly and rearrangement of the postulated supercontinent Rodinia. Paleomagnetic study of rocks from Laurentia have led to a series of paleomagnetic poles that form an apparent polar wander path (APWP) known as the ``Logan Loop" for the older high latitude poles at its apex that continues into the ``Keweenawan Track" of younger lower latitude poles that form a progression as the APWP heads toward the ``Grenville Loop" \citep{Swanson-Hysell2019a}. In particular, recent advances in pairing high-precision zircon U-Pb geochronology with high-quality paleomagnetic poles significantly improved the resolution of the Keweenawan Track and revealed Laurentia's rapid motion ($>$20 cm/yr) during the Midcontinent Rift magmatic activity from ca. 1110 Ma to ca. 1085 Ma \citep{Swanson-Hysell2019a}. 

As the Midcontinent Rift magmatic activity wanes its strength and evolved into a failed intracontinental rift, Laurentia experienced a long period of magmatic quiescence in its interior, where basin subsidence dominated the rift basin. This was followed by the Grenville Orogeny, causing the the inversion of the rift along thrust belts such as the Keweenaw Fault (Fig. xxx). The lack of extensive magmatism from ca.1080 Ma to ca. 980 Ma in the Laurentia interior resulted in a significant gap (for at least 50 myr) of paleomagnetic poles between the Keweenawan Track (developed from interior Laurentia igneous and sedimentary rocks) and the Grenville Loop (poles developed from the Grenville Province rocks) (Fig. xxx; \cite{Swanson-Hysell2019a}).  

Due to the lack of pmag records from the interior Laurentia, various hypotheses for the spatial relationship between the end of the Keweenawan Track and the Grenville Loop have been debated. One model argues for that the difference in pole locations indicates that the Grenville Province was once thousands of kilometers away from Interior Laurentia and was progressively approaching the southern margin of Laurentia in the late Mesoproterozoic \citep{Halls2015}. A corollary of this hypothesis is that there had to be a 4000 km shortening during the Grenville orogeny - colliding Amazonia-Laurentia. The other model favors that the Grenville Loop is a continuation of the Keweenawan Track - that the southerly continental drift of Laurentia continued across the equator onto the southern hemisphere from ca. 1070 Ma to ca.1000 Ma. and  approached this problem by developing data from metamorphosed dike and anorthosite and argued for Grenville  

Paleomagnetic poles from the interior of Laurentia during the quiescence magmatism period is thus crucial for reconcile the discordant paleomagnetic poles of the Keweenawan Track and the Grenville Loop. The
%Q: what are you using the poles from rocks that we know are from the Laurentia interior? (outside of the Grenville package?)

This new age constraint on the Jacobsville provides crucial insight into the tectonic evolution of the Midcontinent Rift and the inversion of the extensional region into a compressional region due to the Grenville orogen. It also makes the Jacobsville sandstone a unique opportunity to investigate Laurentia's paleogeographical position at ca. 1 Ga as a long period of magmatic quiescence for $\sim$50 myr precedes Jacobsville deposition left a gap in Laurentia apparent polar wander path between the Keweenawan Track and the Grenville Loop (Fig. xxx). 

Jacobsville sedimentation was preceded by a long period of volcanic and tectonic quiescence and cratonic stability so that bedrock surfaces became blanketed by paleosol and a surface of chemically resistant debris was dominated by quartz and iron-formation. 

Previous paleomagnetic study on the red beds within the Jacobsville by \cite{Roy1978a} observed different components using AF, thermal, and chemical demagnetizations. Recognized dual polarities and recognized the complex remanence acquisition of red beds which lead to three isolated remanence components from the Jacobsville. However, that study did not recognize the problem of inclination shallowing and did not apply correction factor on the results. Moreover, the course demagnetization steps during thermal demagnetization would have had limited resolution for distinguishing the depositional remanent magnetization (DRM) and secondary chemical remanent magnetization (CRM) which have recently been shown to unblock over a narrow temperature window \cite{Swanson-Hysell2019b}. 

In this study, we use fine-grained to silt-sized, hematite rich red beds from xxx sections along the Keweenaw Peninsula, Michigan with the goal of obtaining high-quality paleomagnetic pole from the jacobsville formation that is corrected for inclination shallowing problem and to investigate the connection between the Keweenawan Track and the Grenville Loop of the Laurentia APWP during the late Mesoproterozoic to the Neoproterozoic. 

\section*{GEOLOGIC SETTINGS}


The Jacobsville sandstone is 
\cite{Kalliokoski1982a} In the Lake Superior syncline the Jacobsville Sandstone is a thick (+900m) fluvial sequence of feldspathic and quartzose sandstones, conglomerates, siltstones, and shales, completely devoid of lava flows or cross-cutting dikes. On the north and south sides of Lake Superior, most of the sandstone occurs as inward-dipping, fault-bounded wedges, separated by regional faults from the Oronto and Bayfield Groups of similar red sandstones, situated on the inner side of the syncline. 

With its age being long debated through the past decades, recently detrital zircon work by \cite{Malone2016a} and \Hodgin{2021a} settled that the maximum deposition age of the Jacobsville sandstone  is 980 Ma, indicating the deposition of the Jacobsville predate or the upper Jacobsville overlaps with the Rigolet phase of the Grenville orogeny. The upper age of the Jacobsville is xxx Ma.


Keweenaw Fault motion post dating Jacobsville in the footwall
Youngest DZ from jacobsvill eis about 993 Ma (sandstone creek, about a few hundred meters from the fault). 
directly above pmag sampling at Agate falls grain B geochron date is (z2, s226, about 1000 Ma grain)

jacobsville (985.5 from calcite in the fault and <993 from zircons) 

Blake's current tectonic model: after MCR volcanics ca. 1084 Ma emplacement follows the deposition of the sediments (Oronto Group) copper harbor conglomerate, Nonesuch formation and the Freda formation. Then the Grenville orogenesis took off with the Ottawan phase from 1020 Ma - 10?? Ma and the Rigolet phase from 995-980 Ma. Paleosol development and the unconformity between the Jacobsville and the volcanics indicate a prolonged period of gap during which the Oronto Group in the backbulge basin where they were deposited were lost, resulting in younger Jacobsville deposition directly on top of the volcanics. The fact that in the Jacobsville has conglomerate and sandstone and siltstone facies in them supports sediment input from the eroded Oronto Group. This indicate the older boundary of the Jacobsville deposition might be closer to the Rigolet Phase than the Ottawan phase. The younger age boundary of the Jacobsville is well defined - given that the maximum deposition age from the zircons of the Jacobsville is 993 Ma and the calcite on the Keweenaw Fault is 985.5 Ma, the deposition of at least the upper portion of the Jacobsville is well constrained to be between 993 and 985 Ma. The deposition of Jacobsville in the back bulge in Laurentia interior is related to the far-field compression caused by the Grenville Orogeny. 

This makes the Jacobsville sandstone one unique opportunity to acquire a paleomagnetic pole position within the gap between the 

\subsection{Jacobsville stratigraphy}

Agate Falls

Natural Wall

Snake Creek

JK creek


\cite{Roy1978a} sampled 37 sites across a lateral distance of 



%\begin{figure}[!ht]
%\noindent\includegraphics[width=\textwidth]{}
%\centering
%\caption{\small{}}
%\label{fig:map}
%\end{figure}

\section*{METHODS and RESULTS}



\section*{DISCUSSION}





\subsection*{ACKNOWLEDGEMENTS}
%We thank James Pierce for his field assistance. 
%\footnotesize


\singlespacing
\clearpage
\bibliographystyle{gsabull}
\bibliography{YZ_ref}

\end{document}